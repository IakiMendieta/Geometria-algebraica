\documentclass{article}
\usepackage{amsthm}
\usepackage{amssymb}
\newtheorem{definicion}{Definición}
\newtheorem{teorema}{Teorema}
\newtheorem{corolario}{Corolario}
\newtheorem{lema}{Lema}
\newtheorem{observacion}{Observacion}

\title{Espectro de un anillo}
\author{Levia MN}
\date{\today}

\begin{document}
\maketitle
\section{Topología}
\begin{definicion}
    Sea A un anillo denotamos $$Spec(A)=\{\mathfrak{p} \subset A; \mathfrak{p} \mbox{ es un ideal primo} \}$$
    Si $M \subset A$ entonces denotamos $$V(M) = \{\mathfrak{p} \in A; M \subset \mathfrak{p}\}$$
    Si $M=\{f\}$ entonces escribimos $V(f)$.
\end{definicion}

\begin{observacion}
    Si $\mathfrak{a}$ es el ideal generado por $M \subset A$ entonces
    $$V(M) = V(\mathfrak{a})$$
\end{observacion}

\begin{proof}
    Si $M \subset \mathfrak{p}$ como $\mathfrak{a}$ es el minimo
    ideal que contiene a $M$ entonces $\mathfrak{a} \subset \mathfrak{p}$
    por lo que $V(M) \subset V(\mathfrak{a})$.
    \\
    Por otro lado si $\mathfrak{a} \subset \mathfrak{p}$ como
    $M \subset \mathfrak{a}$ entonces $M \subset \mathfrak{p}$
    por lo que $V(\mathfrak{a}) \subset V(M)$
    \\
    $\therefore V(M) = V(\mathfrak{a})$
\end{proof}

\begin{lema}
    \begin{enumerate}
        \item Si $\mathfrak{a} \subset \mathfrak{b}$ entonces $V(\mathfrak{b}) \subset V(\mathfrak{a})$
        \item $V(0)=Spec(A)$ y $V(1)=\varnothing$
        \item Si $\{\mathfrak{a}_{i} \subset A ; i\in I\}$ es una familia de ideales de A,
        entonces $$V(\bigcup_{i\in I} \mathfrak{a}_i) = V(\sum_{i\in I}\mathfrak{a}_i) = 
        \bigcap_{i\in I} V(\mathfrak{a}_i)$$
        \item Si $\mathfrak{a}$ y $\mathfrak{b}$ son ideales de A, entonces
        $$V(\mathfrak{a}\cap \mathfrak{b}) = V(\mathfrak{a} \mathfrak{b}) = V(\mathfrak{a})\cup V(\mathfrak{b}) $$
    \end{enumerate}
\end{lema}
\begin{proof}
    \begin{enumerate}
        \item Si $\mathfrak{b} \subset \mathfrak{p}$ entonces $\mathfrak{a} \subset \mathfrak{p}$
    
        \item Para cualquier $\mathfrak{p} \in Spec(A)$ se cumple que $0 \in \mathfrak{p}$
        $$\therefore V(0)=Spec(A)$$
        \\
        Para cualquier $\mathfrak{p} \in Spec(A)$ se cumple que $1 \notin \mathfrak{p}$
        $$\therefore V(1)=\varnothing$$

        \item Por la observación se sigue la primer igualdad, para la segunda 
        Dado $j\in I $ 
        $$\mathfrak{a}_j \subset \bigcup_{i\in I} \mathfrak{a}_i$$ así que por 1.
        $V(\bigcup_{i\in I}\mathfrak{a}_i) \subset V(\mathfrak{a}_j)$
        $$\therefore V(\bigcup_{i\in I}\mathfrak{a}_i) \subset \bigcap_{i \in I} V(\mathfrak{a}_i)$$
        \\
        Por otro lado si $\mathfrak{p} \in \bigcap_{i \in I} V(\mathfrak{a}_i)$ entonces
        para todo $i\in I$ ocurre que $\mathfrak{a}_i \subset \mathfrak{p}$\\
        por lo que $\bigcup_{i\in I} \mathfrak{a}_i \subset \mathfrak{p}$
        $$\therefore \bigcap_{i \in I} V(\mathfrak{a}_i) \subset V(\bigcup_{i\in I}\mathfrak{a}_i)$$

        
        \item Como $\mathfrak{a} \mathfrak{b} \subset \mathfrak{a}\cap \mathfrak{b}\subset \mathfrak{a}, \mathfrak{b}$ entonces por 1.
        $$V(\mathfrak{a}) \cup V(\mathbf{b}) \subset V(\mathfrak{a}\cap \mathfrak{b}) \subset V(\mathfrak{a} \mathfrak{b})$$
        Si $\mathfrak{p} \in V(\mathfrak{a} \mathfrak{b})$ y $\mathfrak{p}\notin V(\mathfrak{a})$ entonces $\mathfrak{a} \not\subset \mathfrak{p}$,
        es decir, existe $a\in \mathfrak{a} \setminus \mathfrak{p}$ pero para toda $b \in \mathfrak{b}$ ocurre que
        $ab \in \mathfrak{a} \mathfrak{b} \subset \mathfrak{p}$ y como $\mathfrak{p}$ es in ideal primo $b \in \mathfrak{p}$
        $$\therefore \mathfrak{b} \subset \mathfrak{p}$$ i.e. $\mathfrak{p} \in V(\mathfrak{b})$
        $$\therefore V(\mathfrak{a} \mathfrak{b}) \subset V(\mathfrak{a}) \cup V(\mathfrak{b})$$
        Lo que prueba la igualdad entre los tres términos.
    \end{enumerate}
\end{proof}

\end{document}