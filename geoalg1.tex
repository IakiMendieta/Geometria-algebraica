\documentclass{article}
\usepackage{amsthm}
\usepackage{amssymb}
\usepackage[spanish, english]{babel}
\selectlanguage{spanish}
\newtheorem{definicion}{Definición}
\newtheorem{teorema}{Teorema}
\newtheorem{corolario}{Corolario}
\newtheorem{lema}{Lema}
\newtheorem{observacion}{Observacion}

\title{Espectro de un anillo}
\author{Levia MN}
\date{\today}

\begin{document}
\maketitle
\section{Topología}
\begin{definicion}
    Sea A un anillo denotamos $$Spec(A)=\{\mathfrak{p} \subset A; \mathfrak{p} \mbox{ es un ideal primo} \}$$
    Si $M \subset A$ entonces denotamos $$V(M) = \{\mathfrak{p} \in A; M \subset \mathfrak{p}\}$$
    Si $M=\{f\}$ entonces escribimos $V(f)$.
\end{definicion}

\begin{observacion}
    Si $\mathfrak{a}$ es el ideal generado por $M \subset A$ entonces
    $$V(M) = V(\mathfrak{a})$$
\end{observacion}

\begin{proof}
    Si $M \subset \mathfrak{p}$ como $\mathfrak{a}$ es el minimo
    ideal que contiene a $M$ entonces $\mathfrak{a} \subset \mathfrak{p}$
    por lo que $V(M) \subset V(\mathfrak{a})$.
    \\
    Por otro lado si $\mathfrak{a} \subset \mathfrak{p}$ como
    $M \subset \mathfrak{a}$ entonces $M \subset \mathfrak{p}$
    por lo que $V(\mathfrak{a}) \subset V(M)$
    \\
    $\therefore V(M) = V(\mathfrak{a})$
\end{proof}

\begin{lema}
    \begin{enumerate}
        \item Si $\mathfrak{a} \subset \mathfrak{b}$ entonces $V(\mathfrak{b}) \subset V(\mathfrak{a})$
        \item $V(0)=Spec(A)$ y $V(1)=\varnothing$
        \item Si $\{\mathfrak{a}_{i} \subset A ; i\in I\}$ es una familia de ideales de A,
        entonces $$V(\bigcup_{i\in I} \mathfrak{a}_i) = V(\sum_{i\in I}\mathfrak{a}_i) = 
        \bigcap_{i\in I} V(\mathfrak{a}_i)$$
        \item Si $\mathfrak{a}$ y $\mathfrak{b}$ son ideales de A, entonces
        $$V(\mathfrak{a}\cap \mathfrak{b}) = V(\mathfrak{a} \mathfrak{b}) = V(\mathfrak{a})\cup V(\mathfrak{b}) $$
    \end{enumerate}
\end{lema}
\begin{proof}
    \begin{enumerate}
        \item Si $\mathfrak{b} \subset \mathfrak{p}$ entonces $\mathfrak{a} \subset \mathfrak{p}$
    
        \item Para cualquier $\mathfrak{p} \in Spec(A)$ se cumple que $0 \in \mathfrak{p}$
        $$\therefore V(0)=Spec(A)$$
        \\
        Para cualquier $\mathfrak{p} \in Spec(A)$ se cumple que $1 \notin \mathfrak{p}$
        $$\therefore V(1)=\varnothing$$

        \item Por la observación se sigue la primer igualdad, para la segunda 
        Dado $j\in I $ 
        $$\mathfrak{a}_j \subset \bigcup_{i\in I} \mathfrak{a}_i$$ así que por 1.
        $V(\bigcup_{i\in I}\mathfrak{a}_i) \subset V(\mathfrak{a}_j)$
        $$\therefore V(\bigcup_{i\in I}\mathfrak{a}_i) \subset \bigcap_{i \in I} V(\mathfrak{a}_i)$$
        \\
        Por otro lado si $\mathfrak{p} \in \bigcap_{i \in I} V(\mathfrak{a}_i)$ entonces
        para todo $i\in I$ ocurre que $\mathfrak{a}_i \subset \mathfrak{p}$\\
        por lo que $\bigcup_{i\in I} \mathfrak{a}_i \subset \mathfrak{p}$
        $$\therefore \bigcap_{i \in I} V(\mathfrak{a}_i) \subset V(\bigcup_{i\in I}\mathfrak{a}_i)$$

        
        \item Como $\mathfrak{a} \mathfrak{b} \subset \mathfrak{a}\cap \mathfrak{b}\subset \mathfrak{a}, \mathfrak{b}$ entonces por 1.
        $$V(\mathfrak{a}) \cup V(\mathbf{b}) \subset V(\mathfrak{a}\cap \mathfrak{b}) \subset V(\mathfrak{a} \mathfrak{b})$$
        Si $\mathfrak{p} \in V(\mathfrak{a} \mathfrak{b})$ y $\mathfrak{p}\notin V(\mathfrak{a})$ entonces $\mathfrak{a} \not\subset \mathfrak{p}$,
        es decir, existe $a\in \mathfrak{a} \setminus \mathfrak{p}$ pero para toda $b \in \mathfrak{b}$ ocurre que
        $ab \in \mathfrak{a} \mathfrak{b} \subset \mathfrak{p}$ y como $\mathfrak{p}$ es in ideal primo $b \in \mathfrak{p}$
        $$\therefore \mathfrak{b} \subset \mathfrak{p}$$ i.e. $\mathfrak{p} \in V(\mathfrak{b})$
        $$\therefore V(\mathfrak{a} \mathfrak{b}) \subset V(\mathfrak{a}) \cup V(\mathfrak{b})$$
        Lo que prueba la igualdad entre los tres términos.
    \end{enumerate}
\end{proof}

\begin{observacion}
    El lema anterior prueba que $\{V(\mathfrak{a}); \mathfrak{a} \subset A \mbox{ es in ideal}\}$
    forman los cerrados de una topología sobre $Spec(A)$
\end{observacion}

Por otro lado vamos a definir el otro lado de la conección de 
Galois que nos gustaría crear.

\begin{definicion}
    Sea $Y \subset Spec(A)$ definimos
    $$I(Y) = \bigcap_{\mathfrak{p}\in Y} \mathfrak{p}$$
    Y definiremos $I(\varnothing) = A$
\end{definicion}

Probemos un lema técnico
\begin{lema}
    Sea $J \subset A$ un ideal entonces
    $rad(J) = \bigcap_{J \subset\mathfrak{p}} \mathfrak{p}$.
    Con $\mathfrak{p} \in Spec(A)$
\end{lema}
\begin{proof}
    Por un lado si $a \in rad(J)$ entonces existe $n\in \mathbb{N}$
    tal que $a^n \in J$ Asi que dado $\mathfrak{p}\in Spec(A)$ tal que
    $J \subset \mathfrak{p}$ se tiene que $a^n \in \mathfrak{p}$ y 
    es un ideal primo se sigue que $a \in \mathfrak{p}$
    $$\therefore rad(J)\subset \bigcap_{J\subset \mathfrak{p}} \mathfrak{p}$$
    \\
    Por otro lado sea $a\in \bigcap_{J\subset \mathfrak{p}} \mathfrak{p}$ y
    supongamos que $a \notin rad(J)$ entonces para cualquier $n \in \mathbb{N}$
    ocurre que $a^n \notin J$ 
    $$\therefore \mathcal{F} = \{I\subset A; 
    \mbox{ I es un ideal, } J\subset I, \quad 
    \forall n\in \mathbb{N}, a^n \notin I\} \neq \varnothing$$
    pues $rad(J) \in \mathcal{F}$ y además esta familia 
    esta ordenada por la contención así que por principio 
    de maximalidad de Hausdorff existe $\mathfrak{q} \in \mathcal{F}$
    maximal tal que $rad(J) \subset \mathfrak{q}$.\\
    Veamos que $\mathfrak{q} \in Spec(A)$, en efecto,
    si $xy \in \mathfrak{q}$ y además se tuviera que $x,y\notin\mathfrak{q}$
    entonces $\mathfrak{q}\subset \mathfrak{q}+ \langle x \rangle, \mathfrak{q}+ \langle y \rangle$
    y por la maximalidad de $\mathfrak{q}$ se tiene que 
    $\mathfrak{q}+ \langle x \rangle, \mathfrak{q}+ \langle y \rangle \notin \mathcal{F}$
    por lo que existen $n, m \in \mathbb{N}$ tales que 
    $a^n \mathfrak{q}+ \langle x \rangle$ y $a^m \in\mathfrak{q}+ \langle y \rangle$
    ya que $J \subset \mathfrak{q} \subset \mathfrak{q}+ \langle x \rangle, \mathfrak{q}+ \langle y \rangle$
    Por lo que $a^n = q_1 + r_1x$ y $a^m = q_2 + r_2 y$ con $q_1, q_2 \in \mathfrak{q}$
    y $r_1, r_2 \in A$
    $$a^{n+m} = (q_1+r_1x)(q_2+r_2y) = q_1q_2 + q_1r_2y + q_2r_1x + r_1r_2xy \in \mathfrak{q}$$
    pues cada termino esta en $\mathfrak{q}$
    lo cual es una contradicción y por ende $\mathfrak{q}$ es un ideal 
    primo, es decir, $\mathfrak{q} \in Spec(A)$.
    Por lo que $a \in \bigcap_{J\subset \mathfrak{p}} \mathfrak{p},
    \subset \mathfrak{q}$  pues $\mathfrak{p} \in Spec(A)$
    lo cual es una contradicción pues $\mathfrak{q}\in \mathcal{F}$.
    Esta contradicción vino de suponer que $a\notin rad(J)$ por lo que 
    $a \in rad(J)$.
    $$\therefore \bigcap_{J\subset \mathfrak{p}} \mathfrak{p} \subset rad(J)$$
    Lo cual da la igual que buscabamos.

\end{proof}

Además tenemos los siguientes resultados

\begin{lema}
    \begin{enumerate}
    \item Si $Y \subset X$ entonces $I(X) \subset I(Y)$
    \item $rad(I(Y)) = I(Y)$
    \item $I(V(\mathfrak{a})) = rad(\mathfrak{a}) $
    y $V(I(Y)) = cl(Y)$ donde cl es la cerradura en Spec(A)
    \end{enumerate}
\end{lema}
\begin{proof}
    \begin{enumerate}
        \item Si $p \in I(X) = \bigcap_{\mathfrak{p}\in X}\mathfrak{p}$
        entonces, para todo $\mathfrak{p}\in X$, se tiene que $p\in \mathfrak{p}$,
        como $Y \subset X$ entonces en particular para todo $\mathfrak{p}\in Y$
        ocurre que $p \in \mathfrak{p}$.
        $$\therefore I(X) \subset I(Y)$$
        \item Siempre ocurre que $I(Y)\subset rad(I(Y))$, y por otro lado
        si $a \in rad(I(Y))$ entonces existe $n\in \mathbb{N}$ tal que $a^n \in I(Y)$
        por lo que $a^n \in \mathfrak{p}$ para cualquier $\mathbf{p}\in Y$. 
        Como cada uno de estos ideales es primo entonces $a \in \mathfrak{p}$ para cualquier 
        $\mathfrak{p}\in Y$.
        $$\therefore rad(I(Y)) \subset I(Y)$$
        Lo que prueba la igualdad
        \item Para lo primero notemos que
        $$I(V(\mathfrak{a}))=\bigcap_{\mathfrak{p}\in V(\mathfrak{a})} \mathfrak{p}
        = \bigcap_{\mathfrak{a}\subset \mathfrak{p}} \mathfrak{p}
        = rad(\mathfrak{a})$$
        Por el lema previo.
        \\
        Por otro lado observemos que $V(I(Y))$ es un cerrado de $Spec(A)$
        Además dado $y\in Y$ entonces $I(Y) = \bigcap_{\mathfrak{p}\in Y} \mathfrak{p}
        \subset y $ por lo que $y \in V(y) \subset (V(I(Y)))$
        $$\therefore Y\subset V(I(Y))$$
        Además si tenemos $V(J)$ un cerrado tal que $Y \subset V(J)$ entonces
        $J \subset y$ para cualquier $y \in Y$ por lo que $J \subset \bigcap_{y \in Y} y = I(Y)$
        asi que $V(I(Y)) \subset V(J)$.
        Lo que lo hace el cerrado más pequeño en contener a $Y$, es decir,
        su cerrad.
        $$\therefore V(I(Y)) = cl(Y)$$
    \end{enumerate}
\end{proof}

\begin{observacion}
    Sean $\mathcal{A} = \{I \subset A; \mbox{ I es un idel radical de A}\}$
    y $\mathcal{B} = \{Y \subset Spec(A); \mbox{ Y es un cerrado de Spec(A)}\}$.\\
    La prueba anterior nos dice que $V:\mathcal{A} \rightarrow \mathcal{B}$ y 
    $I: \mathcal{B} \rightarrow \mathcal{A}$ son inversas una de otra y por ende
    biyecciones entre $\mathcal{A}$ y $\mathcal{B}$.
\end{observacion}

\begin{observacion}
    Si $x\in Spec(A)$ entonces $I({x})=x$, por lo que
    $$V(x)=V(I(\{x\}))=cl(\{x\})$$
\end{observacion}
\end{document}