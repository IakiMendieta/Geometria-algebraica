\documentclass[14pt]{extarticle}
\usepackage{amsthm}
\usepackage{amssymb}
\usepackage{xcolor}
\usepackage{tikz-page}
\usepackage{tikz,tkz-tab,amsmath}
\usepackage[many]{tcolorbox}
\usetikzlibrary{shadows.blur}
\usepackage[spanish]{babel}
\usepackage{geometry}
\geometry{
    a4paper,
    total={170mm,257mm},
    left=20mm,
    top=20mm,
    }

\selectlanguage{spanish}
\tcbuselibrary{theorems}
\newtcbtheorem[number within=section]{teorema}{Teorema}%
{breakable,enhanced,colback=green!5,colframe=green!35!black,fonttitle=\bfseries}{th}
\newtcbtheorem[number within=section]{proposicion}{Proposición}%
{breakable,enhanced,colback=green!5,colframe=green!35!black,fonttitle=\bfseries}{pr}
\newtcbtheorem[number within=section]{corolario}{Corolario}%
{breakable,enhanced,colback=green!5,colframe=green!35!black,fonttitle=\bfseries}{cr}
\newtcbtheorem[number within=section]{lema}{Lema}%
{breakable,enhanced,colback=green!5,colframe=green!35!black,fonttitle=\bfseries}{lm}
\newtcbtheorem[number within=section]{definicion}{Definición}%
{breakable,enhanced,colback=red!5,colframe=red!35!black,fonttitle=\bfseries}{df}
\newtcbtheorem[number within=section]{observacion}{Observación}%
{breakable,enhanced,colback=blue!5,colframe=blue!35!black,fonttitle=\bfseries}{ob}
\newtcbtheorem[number within=section]{ejemplo}{Ejemplo}%
{breakable,enhanced,colback=blue!5,colframe=blue!35!black,fonttitle=\bfseries}{ej}

\tikzset{
        secnode/.style={
                minimum height=1cm,
                inner xsep=20pt,
                rotate=90,
                anchor=north east,
                draw=white,
                fill=violet,
                text=white,
                blur shadow={shadow blur steps=5,shadow blur extra rounding=1.3pt}},
        pagenode/.style={
                minimum width=5mm,
                minimum height=1cm,
                inner sep=2pt,
                anchor=south east,
                draw=white,
                fill=violet,
                text=white,
                blur shadow={shadow blur steps=5,shadow blur extra rounding=1.3pt}}
        }
\newcommand{\tikzpagelayout}{
        \draw[violet,line width=2pt,rounded corners=20pt] ([xshift=5mm]page.northwest) |- ([xshift=-2cm,yshift=5mm]page.southeast);
        \node[secnode] at (page.northwest) {Geometría algebraica. \thesection};
        \node[pagenode] at ([xshift=-1cm]page.southeast) {\thepage};
      }

\pagestyle{plain}

\title{Espectro de un anillo}
\author{Levia MN}
\date{\today}

\begin{document}
\maketitle
\section{Topología}

\begin{definicion}{Espectro de un anillo.}{}
    Sea A un anillo denotamos 
    $$Spec(A)=\{\mathfrak{p} \subset A \mid \mathfrak{p} \mbox{ es un ideal primo} \}$$
    Si $M \subset A$ entonces denotamos 
    $$V(M) = \{\mathfrak{p} \in Spec(A) \mid M \subset \mathfrak{p}\}$$
    Si $M=\{f\}$ entonces escribimos $V(f)$.
\end{definicion}

\begin{observacion}{}{}
    Si $\mathfrak{a}$ es el ideal generado por $M \subset A$ entonces
    $$V(M) = V(\mathfrak{a})$$
\end{observacion}
\begin{proof}
    \begin{enumerate}
        \item[$\subset$)]Si $M \subset \mathfrak{p}$ como $\mathfrak{a}$ es el minimo
        ideal que contiene a $M$ entonces $\mathfrak{a} \subset \mathfrak{p}$
        por lo que $V(M) \subset V(\mathfrak{a})$.
        \item[$\supset$)]Por otro lado si $\mathfrak{a} \subset \mathfrak{p}$ como
        $M \subset \mathfrak{a}$ entonces $M \subset \mathfrak{p}$
        por lo que $V(\mathfrak{a}) \subset V(M)$.
    \end{enumerate}
    $\therefore V(M) = V(\mathfrak{a})$
\end{proof}

\begin{lema}{}{}
    \begin{enumerate}
        \item Si $\mathfrak{a} \subset \mathfrak{b}$ entonces $V(\mathfrak{b}) \subset V(\mathfrak{a})$
        \item $V(0)=Spec(A)$ y $V(1)=\varnothing$
        \item Si $\{\mathfrak{a}_{i} \subset A ; i\in I\}$ es una familia de ideales de A, entonces 
        $$V\left( \bigcup_{i\in I} \mathfrak{a}_i\right) 
        = V\left( \sum_{i\in I}\mathfrak{a}_i\right) 
        = \bigcap_{i\in I} V(\mathfrak{a}_i)$$
        \item Si $\mathfrak{a}$ y $\mathfrak{b}$ son ideales de A, entonces
        $$V(\mathfrak{a}\cap \mathfrak{b}) = V(\mathfrak{a} \mathfrak{b}) = V(\mathfrak{a})\cup V(\mathfrak{b}) $$
    \end{enumerate}
\end{lema}
\begin{proof}
    \begin{enumerate}
        \item Si $\mathfrak{b} \subset \mathfrak{p}$ entonces $\mathfrak{a} \subset \mathfrak{p}$
    
        \item Para cualquier $\mathfrak{p} \in Spec(A)$ se cumple que $0 \in \mathfrak{p}$
        $$\therefore V(0)=Spec(A)$$
        \\
        Para cualquier $\mathfrak{p} \in Spec(A)$ se cumple que $1 \notin \mathfrak{p}$
        $$\therefore V(1)=\varnothing$$

        \item Por la observación se sigue la primer igualdad, para la segunda 
        Dado $j\in I $ 
        $$\mathfrak{a}_j \subset \bigcup_{i\in I} \mathfrak{a}_i$$ así que por 1.
        $V(\bigcup_{i\in I}\mathfrak{a}_i) \subset V(\mathfrak{a}_j)$
        $$\therefore V(\bigcup_{i\in I}\mathfrak{a}_i) \subset \bigcap_{i \in I} V(\mathfrak{a}_i)$$
        \\
        Por otro lado si $\mathfrak{p} \in \bigcap_{i \in I} V(\mathfrak{a}_i)$ entonces
        para todo $i\in I$ ocurre que $\mathfrak{a}_i \subset \mathfrak{p}$\\
        por lo que $\bigcup_{i\in I} \mathfrak{a}_i \subset \mathfrak{p}$
        $$\therefore \bigcap_{i \in I} V(\mathfrak{a}_i) \subset V(\bigcup_{i\in I}\mathfrak{a}_i)$$

        
        \item Como $\mathfrak{a} \mathfrak{b} \subset \mathfrak{a}\cap \mathfrak{b}\subset \mathfrak{a}, \mathfrak{b}$ entonces por 1.
        $$V(\mathfrak{a}) \cup V(\mathfrak{b}) \subset V(\mathfrak{a}\cap \mathfrak{b}) \subset V(\mathfrak{a} \mathfrak{b})$$
        Si $\mathfrak{p} \in V(\mathfrak{a} \mathfrak{b})$ y $\mathfrak{p}\notin V(\mathfrak{a})$ entonces $\mathfrak{a} \not\subset \mathfrak{p}$,
        es decir, existe $a\in \mathfrak{a} \setminus \mathfrak{p}$ pero para toda $b \in \mathfrak{b}$ ocurre que
        $ab \in \mathfrak{a} \mathfrak{b} \subset \mathfrak{p}$ y como $\mathfrak{p}$ es in ideal primo $b \in \mathfrak{p}$
        $$\therefore \mathfrak{b} \subset \mathfrak{p}$$ i.e. $\mathfrak{p} \in V(\mathfrak{b})$
        $$\therefore V(\mathfrak{a} \mathfrak{b}) \subset V(\mathfrak{a}) \cup V(\mathfrak{b})$$
        Lo que prueba la igualdad entre los tres términos.
    \end{enumerate}
\end{proof}

\begin{observacion}{}{}
    El lema anterior prueba que $\{V(\mathfrak{a}); \mathfrak{a} \subset A \mbox{ es in ideal}\}$
    forman los cerrados de una topología sobre $Spec(A)$
\end{observacion}

Por otro lado vamos a definir el otro lado de la conección de 
Galois que nos gustaría crear.

\begin{definicion}{Ideal asociado}{}
    Sea $Y \subset Spec(A)$ definimos
    $$I(Y) = \bigcap_{\mathfrak{p}\in Y} \mathfrak{p}$$
    Y definiremos $I(\varnothing) = A$
\end{definicion}

Probemos un lemma técnico
\begin{lema}{}{}
    Sea $J \subset A$ un ideal entonces
    $rad(J) = \bigcap_{\mathfrak{p} \in V(J)} \mathfrak{p}$.
\end{lema}
\begin{proof}
    Por un lado si $a \in rad(J)$ entonces existe $n\in \mathbb{N}$
    tal que $a^n \in J$ Asi que dado $\mathfrak{p}\in Spec(A)$ tal que
    $J \subset \mathfrak{p}$ se tiene que $a^n \in \mathfrak{p}$ y 
    es un ideal primo se sigue que $a \in \mathfrak{p}$
    $$\therefore rad(J)\subset \bigcap_{\mathfrak{p} \in V(J)} \mathfrak{p}$$
    \\
    Por otro lado sea $a\in \bigcap_{\mathfrak{p} \in V(J)} \mathfrak{p}$ y
    supongamos que $a \notin rad(J)$ entonces para cualquier $n \in \mathbb{N}$
    ocurre que $a^n \notin J$ 
    $$\therefore \mathcal{F} = \{I\subset A \mid 
    \mbox{ I es un ideal, } J\subset I, \quad 
    \forall n\in \mathbb{N}, a^n \notin I\} \neq \varnothing$$
    pues $rad(J) \in \mathcal{F}$ y además esta familia 
    esta ordenada por la contención así que por principio 
    de maximalidad de Hausdorff existe $\mathfrak{q} \in \mathcal{F}$
    maximal tal que $rad(J) \subset \mathfrak{q}$.\\
    Veamos que $\mathfrak{q} \in Spec(A)$, en efecto,
    si $xy \in \mathfrak{q}$ y además se tuviera que $x,y\notin\mathfrak{q}$
    entonces $\mathfrak{q}\subset \mathfrak{q}+ \langle x \rangle, \mathfrak{q}+ \langle y \rangle$
    y por la maximalidad de $\mathfrak{q}$ se tiene que 
    $\mathfrak{q}+ \langle x \rangle, \mathfrak{q}+ \langle y \rangle \notin \mathcal{F}$
    por lo que existen $n, m \in \mathbb{N}$ tales que 
    $a^n \mathfrak{q}+ \langle x \rangle$ y $a^m \in\mathfrak{q}+ \langle y \rangle$
    ya que $J \subset \mathfrak{q} \subset \mathfrak{q}+ \langle x \rangle, \mathfrak{q}+ \langle y \rangle$
    Por lo que $a^n = q_1 + r_1x$ y $a^m = q_2 + r_2 y$ con $q_1, q_2 \in \mathfrak{q}$
    y $r_1, r_2 \in A$
    $$a^{n+m} = (q_1+r_1x)(q_2+r_2y) = q_1q_2 + q_1r_2y + q_2r_1x + r_1r_2xy \in \mathfrak{q}$$
    pues cada término esta en $\mathfrak{q}$
    lo cual es una contradicción y por ende $\mathfrak{q}$ es un ideal 
    primo, es decir, $\mathfrak{q} \in Spec(A)$.
    Por lo que $a \in \bigcap_{\mathfrak{p} \in V(J)} \mathfrak{p},
    \subset \mathfrak{q}$  pues $\mathfrak{p} \in Spec(A)$
    lo cual es una contradicción pues $\mathfrak{q}\in \mathcal{F}$.
    Esta contradicción vino de suponer que $a\notin rad(J)$ por lo que 
    $a \in rad(J)$.
    $$\therefore \bigcap_{\mathfrak{p} \in V(J)} \mathfrak{p} \subset rad(J)$$
    Lo cual da la igual que buscabamos.
\end{proof}

Además tenemos los siguientes resultados

\begin{lema}{}{}
    \begin{enumerate}
        \item Si $Y \subset X$ entonces $I(X) \subset I(Y)$
        \item $rad(I(Y)) = I(Y)$
        \item $I(V(\mathfrak{a})) = rad(\mathfrak{a}) $
        y $V(I(Y)) = cl(Y)$ donde cl es la cerradura en Spec(A)
        \end{enumerate}
\end{lema}
\begin{proof}
    \begin{enumerate}
        \item Si $p \in I(X) = \bigcap_{\mathfrak{p}\in X}\mathfrak{p}$
        entonces, para todo $\mathfrak{p}\in X$, se tiene que $p\in \mathfrak{p}$,
        como $Y \subset X$ entonces en particular para todo $\mathfrak{p}\in Y$
        ocurre que $p \in \mathfrak{p}$.
        $$\therefore I(X) \subset I(Y)$$
        \item Siempre ocurre que $I(Y)\subset rad(I(Y))$, y por otro lado
        si $a \in rad(I(Y))$ entonces existe $n\in \mathbb{N}$ tal que $a^n \in I(Y)$
        por lo que $a^n \in \mathfrak{p}$ para cualquier $\mathbf{p}\in Y$. 
        Como cada uno de estos ideales es primo entonces $a \in \mathfrak{p}$ para cualquier 
        $\mathfrak{p}\in Y$.
        $$\therefore rad(I(Y)) \subset I(Y)$$
        Lo que prueba la igualdad
        \item Para lo primero notemos que
        $$I(V(\mathfrak{a}))=\bigcap_{\mathfrak{p}\in V(\mathfrak{a})} \mathfrak{p}
        = \bigcap_{\mathfrak{a}\subset \mathfrak{p}} \mathfrak{p}
        = rad(\mathfrak{a})$$
        Por el lemma previo.
        \\
        Por otro lado observemos que $V(I(Y))$ es un cerrado de $Spec(A)$
        Además dado $y\in Y$ entonces $I(Y) = \bigcap_{\mathfrak{p}\in Y} \mathfrak{p}
        \subset y $ por lo que $y \in V(y) \subset (V(I(Y)))$
        $$\therefore Y\subset V(I(Y))$$
        Además si tenemos $V(J)$ un cerrado tal que $Y \subset V(J)$ entonces
        $J \subset y$ para cualquier $y \in Y$ por lo que $J \subset \bigcap_{y \in Y} y = I(Y)$
        asi que $V(I(Y)) \subset V(J)$.
        Lo que lo hace el cerrado más pequeño en contener a $Y$, es decir,
        su cerradura.
        $$\therefore V(I(Y)) = cl(Y)$$
    \end{enumerate}
\end{proof}

\begin{observacion}{}{}
    Sean 
    \begin{align*}
        &\mathcal{A} = \{I \subset A; \mbox{ I es un idel radical de A}\}\\
        &\mathcal{B} = \{Y \subset Spec(A); \mbox{ Y es un cerrado de Spec(A)}\}
    \end{align*}
    La prueba anterior nos dice que $V:\mathcal{A} \rightarrow \mathcal{B}$ y 
    $I: \mathcal{B} \rightarrow \mathcal{A}$ son inversas una de otra y por ende
    biyecciones entre $\mathcal{A}$ y $\mathcal{B}$.
\end{observacion}

\begin{observacion}{}{}
    Si $\mathfrak{p}\in Spec(A)$ entonces
    $I({\mathfrak{p}})=\mathfrak{p}$, 
    por lo que
    $$V(\mathfrak{p})
    =V(I(\{\mathfrak{p}\}))
    =cl(\{\mathfrak{p}\})$$
\end{observacion}

\begin{lema}{}{}
    Si $I, J \subset A$ son ideales de A, entonces 
    $I \subset rad(J)$ si y solo si $rad(I)\subset rad(J)$
\end{lema}
\begin{proof}
    \begin{enumerate}
        \item[$\implies$)]Supongamos que $I \subset rad(J)$ y sea $a \in rad(I)$,
        entonces existe $n \in \mathbb{N}$ tal que $a^n \in I$ por lo que 
        $a^n \in rad(J) = \bigcap_{J\subset \mathfrak{p}}\mathfrak{p}$ donde $\mathfrak{p}\in Spec(A)$
        asi que para todo $\mathfrak{p} \in Spec(A)$ tal que $J\subset \mathfrak{p}$
        se tiene que $a^n \in \mathfrak{p}$ y como cada uno es un ideal primo
        $a\in \mathfrak{p}$.
        $$\therefore a \in \bigcap_{I\subset \mathfrak{p}}\mathfrak{p} = rad(J)$$
        Concluimos que $rad(I) \subset rad(J)$.
        \item[$\impliedby$)]Supongamos que $rad(I)\subset rad(J)$ entonces como 
        $I\subset rad(I)$ se sigue lo que queriamos.
    \end{enumerate}
\end{proof}

\begin{corolario}{}{rad}
    Sea $g \in A$ y $I, J \subset A$ un ideales de A,
    entonces 
    \begin{align*}
        &V(I) \subset V(J) \iff \\
        &rad(J)= \bigcap_{\mathfrak{p} \in V(J)} \mathfrak{p}\subset \bigcap_{\mathfrak{p}\in V(I)}\mathfrak{p} =rad(I)
    \end{align*}
    En particular para $g\in A$
    \begin{align*}
        &V(I) \subset V(g) \iff \\ 
        &\{g\} \subset rad(I) \iff \\
        &g \in rad(I).
    \end{align*}
\end{corolario}

\begin{definicion}{Abiertos principales}{}
    Sea $f\in A$ definimos
    $$D(f) = Spec(A) \setminus V(f)$$
    y notamos que es un abierto, a los abiertos de 
    este tipo se les llama abiertos principales.
\end{definicion}

\begin{observacion}{}{}
    Notemos los siguientes hechos:
    \begin{enumerate}
        \item $D(0)=\varnothing$
        \item $D(1)=Spec(A)$
        \item Dado $\mathfrak{p} \in Spec(A)$ se cumple
        que $fg \notin \mathfrak{p}$ si y solo $f,g \notin \mathfrak{p}$
        pues $\mathfrak{p}$ es un ideal primo, por lo que $D(fg)= D(f)\cap D(g)$
    \end{enumerate}
    
\end{observacion}

\begin{lema}{}{}
    Sean $\{f_i\mid i\in \Lambda\} \subset A$ y $g \in A$,
    entonces $D(g)\subset \bigcup_{i\in \Lambda}D(f_i) $ si y solo si
    $g \in rad \left( \sum_{i\in I} \langle f_i \rangle\right)$
\end{lema}
\begin{proof}
    $$D(g) \subset \bigcup_{i\in \Lambda}D(f_i) = \bigcup_{i\in  \Lambda} Spec(A)\setminus V(f_i)
    = Spec(A)\setminus \left( \bigcap_{i\in  \Lambda}V(f_i)\right) $$
    Y como $\bigcap_{i\in  \Lambda}V(f_i) = V(\sum_{i\in \Lambda}\langle f_i \rangle)$
    entonces $$Spec(A)\setminus \left(\bigcap_{i\in  \Lambda}V(f_i) \right) = Spec(A)\setminus V\left(\sum_{i\in \Lambda}\langle f_i \rangle\right)$$
    por lo que $D(g)\subset \bigcup_{i \in \Lambda}D(f_i)$ si y solo si
    $$V\left(\sum_{i\in \Lambda}\langle f_i \rangle \right) \subset V(g)$$
    y por el lema \ref{cr:rad} esto ocurre si y solo si
    $$g \in rad \left(\sum_{i\in \Lambda}\langle f_i \rangle\right)$$
\end{proof}

\begin{observacion}{}{}
    En particular aplicando el lema \ref{cr:rad} a 
    $g = 1$ entonces:
    \begin{align*}
        D(1) = Spec(A) \subset \bigcup_{i \in \Lambda } D(f_i) \iff\\
        1 \in rad(\sum_{i\in \Lambda} \langle f_i \rangle) \iff \\
        1\in \sum_{i\in \Lambda} \langle f_i \rangle \iff \\
        \langle f_i \mid i\in \Lambda \rangle = A
    \end{align*} 
    en particular $1 = \sum_{j=1}^{n} \alpha_j f_{i_j}$ con $n\in \mathbb{N}, \quad \alpha_j \in A \quad y \quad i_j \in \Lambda$.\\
    Por lo que $\langle f_{i_j}\mid j\in\{1,...,n \} \rangle = A$
    y por ende $Spec(A) \subset \bigcup_{j=1}^{n}D(f_{i_j})$
    Por lo que toda cubierta tiene una subcubierta finita.
\end{observacion}

Hemos probado entonces que 

\begin{corolario}{}{}
    Spec(A) es compacto.
\end{corolario}

\begin{proposicion}{}{}
    Sea A un anillo, entonces
    $\mathcal{B} = \{D(f); f\in A\}$
    es una base para la topología de Spec(A).
\end{proposicion}
\begin{proof}
    Sea $U \subset Spec(A)$ un abierto, entonces
    $Spec(A)\setminus U$ es un conjunto cerrado,
    por lo que existe $I \subset A$ ideal de $A$ tal que 
    $Spec(A)\setminus U = V(I)$, además.
    $$V(I)= V\left( \bigcup_{f\in I} \{f\}\right) = \bigcap_{f\in I}V(f)$$
    De donde se sigue que:
    $$U = Spec(A)\setminus V(I) 
    = Spec(A)\setminus\left( \bigcap_{f\in I}V(f)\right) 
    = \bigcup_{f\in I} (Spec(A)\setminus V(f))
    = \bigcup_{f \in I}D(f)$$
\end{proof}

\begin{proposicion}{}{}
    Sean $A$ un anillo y $f\in A$, entonces
    $D(f) \subset Spec(A)$ es compacto.
\end{proposicion}
\begin{proof}
    Sea $\mathcal{U}=\{D(g_i)\mid i\in I\}$ una cubierta
    de $D(f)$ conformada por abiertos básicos.
    Como son una cubierta $D(f) \subset \bigcup_{i \in I}D(g_i)$
    y por el lema \ref{cr:rad} se sigue que
    $f\in rad\left(\sum_{i\in I}\langle g_i \rangle\right)$.
    Y notemos que
    $$\sum_{i\in I}\langle g_i \rangle
    = \langle \{g_i \mid i\in I \} \rangle$$
    Por lo que existe $n \in \mathbb{N}$ tal que
    $f^n \in \langle\{g_i\mid i\in I\} \rangle$ asi que
    existe $m \in \mathbb{N}$ para la cual se cumple que
    $f^n = \sum_{j=1}^{m} a_j g_{i_j}$ con $a_j \in A$
    e $i_j \in I$ para cada $j \in \{1,\dots,m\}$, en particular,
    $f^n \in \langle \{g_{i_j}\mid j\in \{1,\dots, m\} \} \rangle$
    por lo que, 
    $f \in rad \left( \sum_{j=1}^{m} \langle g_{i_j} \rangle\right)$
    y de nuevo por el lema \ref{cr:rad} se sigue que, 
    $D(f) \subset \bigcup_{j=1}^m D(g_{i_j})$ así que podemos conluir
    que $\{g_{i_j}\mid j\in\{1,\dots,m\}\} \subset \mathcal{U}$ es una
    subcubierta finita.
\end{proof}

\begin{proposicion}{}{}
    Sea A un anillo. Un subespacio $Y \subset Spec(A)$
    es irreducible si y solo si $\mathfrak{p} = I(Y)$
    es un ideal primo, es decir, $I(Y) \in Spec(A)$.
    Más aún en este caso $\{\mathfrak{p}\}$ es denso en 
    $cl(Y)$.
\end{proposicion}
\begin{proof}
    $\implies$ Supongamos que $Y$ es irreducible y sean 
    $f,g \in A$ tales que $fg \in \mathfrak{p}$, entonces
    $$Y \subset cl(Y) = V(I(Y)) = V(\mathfrak{p}) \subset V(fg) = V(f) \cup V(g)$$.
    y como $Y$ es irreducible entonces $Y \subset V(f)$ o $Y \subset V(g)$,
    por lo que $f \in I(V(f)) \subset I(Y) = \mathfrak{p}$
    o $g \in I(V(g)) \subset I(Y) = \mathfrak{p}$, es decir,
    $f \in \mathfrak{p}$ o $g \in \mathfrak{p}$. Lo que prueba
    que $\mathfrak{p}$ es in ideal primo.\\
    $\impliedby$ Supongamos que $\mathfrak{p} \in Spec(A)$
    entonces 
    $$cl(Y) = V(I(Y)) = V(\mathfrak{p}) = V(I(\{\mathfrak{p}\})) = cl(\{\mathfrak{p}\})$$
    y como $\{\mathfrak{p}\}$ es irreducible, $cl(Y)$ es irreducible y por ende
    $Y$ es irreducible y eso también prueba que $\{\mathfrak{p}\}$
    es denso en $cl(Y)$.
\end{proof}

\begin{corolario}{}{}
    Sea $\mathcal{A} = \{Y\subset Spec(A); \mbox{Y es cerrado e irreducible}\}$,
    entonces 
    $$f:Spec(A) \rightarrow \mathcal{A}$$ 
    dada por $f(\mathfrak{p}) = V(\mathfrak{p}) = cl(\mathfrak{p})$ es una biyección.
\end{corolario}
\begin{proof}
    Como $\{\mathfrak{p}\}$ es irreducible entonces
    $f(\mathfrak{p}) = cl(\{\mathfrak{p}\})$ es cerrado e irreducible.
    Y si $Y \subset Spec(A)$ es un cerrado irreducible entonces
    $I(Y) \in Spec(A)$ y además $f(I(Y)) = cl(Y) = Y$ por la Proposición anterior
    por lo que $f$ es sobre, y además si $cl(\{\mathfrak{p}\}) = cl(\{\mathfrak{q}\})$
    entonces $V(\{\mathfrak{p}\}) = V(\{\mathfrak{q}\})$ por lo que $\mathfrak{p} = \mathfrak{q}$.
    Asi que $f$ es inyectiva y por ende biyectiva.
\end{proof}

Notemos además que la correspondencia anterior
invierte el orden de la contención de cada lado.
Pensando a los elementos de $Spec(A)$ con el orden 
dado al ser ideales de $A$. Así que si tenemos un
ideal primo $\mathfrak{p} \subset A$ minimal 
$V(\mathfrak{p}) \subset Spec(A)$ es  un cerrado e 
irreducible maximal por lo que es una componente irreducible 
de $Spec(A)$. Más aún si hay mas de un ideal primo minimal
entonces $Spec(A)$ tendrá mas de una componente irreducible
por lo que $Spec(A)$ es irreducible si y solo si tiene un único
ideal primo minimal, de hecho, eso lo hace mínimo.

\begin{ejemplo}{}{}
    Un ejemplo concreto de esto podría ser $Spec(\mathbb{Z})$
    Aquí nuestro único ideal minimal (de hecho es mínimo) es 
    $\{0\}$ por lo que $V(0) = Spec(\mathbb{Z})$ es un cerrado
    e irreducible. Más aún sus cerrados irreducibles serán de
    la forma $V(\langle p \rangle)$ con $p \in \mathbb{Z}$ un
    número primo y observemos más
    \begin{align*}
        &V(\langle p \rangle)\\ 
        &= \{\langle q \rangle; \langle p \rangle
        \subset \langle q\rangle \mbox{ con q primo}\}\\
        &=\{\langle q \rangle; q \mbox{ es primo y divide a } p\}\\
        &=\{\langle p \rangle\}
    \end{align*} 
    Por lo que todos los demás cerrados irreducibles son unitarios.

\end{ejemplo}

\begin{proposicion}{}{nilpotentes}
    Sea $f\in A$, entonces $D(f) = \varnothing$
    si y solo si $f$ es nilpotente
\end{proposicion}
\begin{proof}
    \begin{align*}
        D(f) = \varnothing \iff \\
        D(f) \subset \varnothing = D(0) \iff \\
        f \in rad(\{0\}) \iff \\
        \mbox{existe }n\in\mathbb{N} \mbox{ tal que }
        f^n = 0 \iff \\
        f \mbox{ es nilpotente.}
    \end{align*}
\end{proof}

\begin{definicion}{Puntos}{puntos}
    Sean $X$ un espacio topologico y $x,y \in X$
    definimos los siguientes conceptos:
    \begin{enumerate}
        \item Decimos que $x$ es un punto cerrado
        si $\{x\}$ es un conjunto cerrado.
        \item Decimos que $x$ es un punto generico
        si $cl(\{x\}) = X$.
        \item Decimos que $x$ es una generalizacion
        de $y$ (o que $y$ es una especialización de $x$)
        si $y \in cl(\{x\})$.
        \item Decimos que $x$ es un punto maximal si 
        $cl(\{x\})$ es una componente irreducible de $X$.
    \end{enumerate}
\end{definicion}

\begin{ejemplo}{}{}
    Si en la definición \ref{df:puntos}
    tenemos que $X= Spec(A)$ para algún
    anillo $A$ entonces las definiciones
    tienen significados algebraicos.
    \begin{enumerate}
        \item $x$ es cerrado si es un 
        ideal maximal de $A$.
        \item $x$ es un punto generico si 
        es (el único) ideal primo mínimo de $A$.
        y notemos que para que eso ocurra entonces
        $x = rad(\{0\})$. Así que podemos conluir que
        la existencia de un  punto generico en $Spec(A)$
        depende de que $rad(\{0\}) \in Spec(A)$.
        \item $x$ es una generalización de $y$
        si $x \subset y$.
        \item $x$ es un punto maximal si $x$ es 
        un ideal primo minimal.
    \end{enumerate}
\end{ejemplo}

\begin{definicion}{Radicales}{}
    Sea $A$ un anillo definimos el nilradical de $A$
    $$Nil(A) = rad(\{0\})$$
    Y el radical de Jacobson de $A$.
    $$Jac(A) = \bigcap_{\mathfrak{m} \in \mathcal{M}} \mathfrak{m}$$
    donde $\mathcal{M} 
    = \{\mathfrak{m}\subset A \mid \mathfrak{m} \mbox{ es un ideal maximal de } A\}$.
\end{definicion}

\begin{proposicion}{}{}
    Sea $A$ un anillo entonces
    $Nil(A)=Jac(A)$ si y solo si 
    para todo $U \subset Spec(A)$ abierto no vacio
    existe $x \in U$ punto cerrado de $Spec(A)$.
\end{proposicion}
\begin{proof}
    \begin{enumerate}
        \item[$\implies$)] Supongamos que $Nil(A) = Jac(A)$
        y sea $U \subset Spec(A)$ un abierto no vacío y supongamos
        que no tiene ningun punto cerrado, es decir 
        $\forall \mathfrak{m} \in \mathcal{M},
        \mathfrak{m} \in Spec(A) \setminus U$.\\
        Primero, como $U$ es abierto entonces existe $I \subset A$
        un ideal de $A$ tal que $U= Spec(A)\setminus V(I)$.
        Entonces nuestra hipótesis se traduce a que 
        $\forall \mathfrak{m} \in \mathcal{M},
        \mathfrak{m} \in V(I)$, es decir, $I \subset \mathfrak{m}$,
        por ende $I \subset Jac(A)= \bigcap_{\mathfrak{m}\in \mathcal{M}}\mathfrak{m}$
        y como $Nil(A)=Jac(A)$, ocurre que $I \subset Nil(A)$.
        Sin embargo, notemos que $Nil(A) = \bigcap_{\mathfrak{q}\in Spec(A)}\mathfrak{q}$ 
        por el lema \ref{cr:rad} asi que para cualquier $\mathfrak{p}\in Spec(A)$,
        $Nil(A) \subset \mathfrak{p}$ por lo que $I \subset \mathfrak{p}$ y entonces $V(I)=Spec(A)$
        pero entonces $U = \varnothing$ ! lo cual es una contradicción.
        Por ende existe $\mathfrak{m}\in \mathcal{M}$ tal que $\mathfrak{m} \in U$.
        \item[$\impliedby$)] Por contapuesta supongamos que $Nil(A)\neq Jac(A)$
        y notemos que como $\mathcal{M} \subset Spec(A)$ entonces $Nil(A) \subset Jac(A)$
        y veamos ahora que:
        $$V(Jac(A))=\{\mathfrak{p}\in Spec(A)\mid Jac(A)\subset \mathfrak{p}\}\neq Spec(A)$$
        pues si fuera $Spec(A)$ entonces para todo $\mathfrak{p}\in Spec(A)$
        $Jac(A)\subset \mathfrak{p}$ y por ende $Jac(A)\subset Nil(A)$!
        Así que $U = Spec(A)\setminus V(Jac(A)) \neq \varnothing$
        sin embargo dado cualquier $\mathfrak{m}\in \mathcal{M}$ se tien que
        $Jac(A)\subset \mathfrak{m}$ por lo que $\mathfrak{m}\in V(Jac(A))$, 
        es decir 
        $$\forall\mathfrak{m}\in\mathcal{M}, \mathfrak{m}\in Spec(A)\setminus U$$
        Por lo tanto $U$ no contiene ningún punto maximal.
    \end{enumerate}
\end{proof}

\begin{observacion}{}{inducido}
    Sean $A, B$ un anillos y $\varphi:A\rightarrow B$ 
    un morfismo de anillos, si $\mathfrak{q}\in Spec(B)$
    entonces $\varphi^{-1}(\mathfrak{q}) \in Spec(A)$.
\end{observacion}
\begin{proof}
    Es claro que $\varphi^{-1}(\mathfrak{q})\subset A$
    es un ideal.
    Sean $f,g \in A$ tales que $fg\in \varphi^{-1}(\mathfrak{q})$
    entonces $\varphi(fg)=\varphi(f)\varphi(g)\in \mathfrak{q}$
    y como $\mathfrak{q}$ es primo entonces $\varphi(f)\in \mathfrak{q}$
    o $\varphi(g)\in \mathfrak{q}$, lo que es equivalente a decir 
    $f \in \varphi^{-1}(\mathfrak{q})$ o $g\in \varphi^{-1}(\mathfrak{q})$.
    Así que $\varphi^{-1}(\mathfrak{q})$ es un ideal primo de $A$.
\end{proof}

La observacion \ref{ob:inducido} nos dice que
todo morfimo de anillo induce un morfismo en
los $Spec$ de los anillos. Definamosla y veamos 
sus propiedades.

\begin{definicion}{}{}
    Dados $A, B$ anillos y $\varphi:A \rightarrow B$
    un morfimo de anillos ente ellos definimos la 
    función $Spec(\varphi):Spec(B)\rightarrow Spec(A)$
    dada por $Spec(\varphi)(\mathfrak{q})=\varphi^{-1}(\mathfrak{q})$
\end{definicion}

Probemos antes un lema técnico

\begin{lema}{}{rad_conmuta}
    Sean $A, B$ anillos, 
    $\mathfrak{b}\subset B$ un ideal de $B$
    y $\varphi:A\rightarrow B$
    un morfimo de anillos entre elllos,
    entonces $rad(\varphi^{-1}(\mathfrak{b}))
    = \varphi^{-1}(rad(\mathfrak{b}))$
\end{lema}
\begin{proof}
    \begin{align*}
        b\in rad(\varphi^{-1}(\mathfrak{b})) \iff\\
        \mbox{existe }n\in\mathbb{N} \mbox{ tal que }
        b^n \in \varphi^{-1}(\mathfrak{b}) \iff \\
        \mbox{existe }n\in\mathbb{N} \mbox{ tal que }
        \varphi(b^n)\in \mathfrak{b} \iff \\
        \mbox{existe }n\in\mathbb{N} \mbox{ tal que }
        \varphi(b)^n \in \mathfrak{b} \iff \\
        \varphi(b)\in rad(\mathfrak{b}) \iff \\
        b \in \varphi^{-1}(rad(\mathfrak{b})).
    \end{align*}
\end{proof}

\begin{proposicion}{}{continuidad}
    Dados $A, B$ anillos, $M\subset A$, $\mathfrak{b} \subset B$
    un ideal y $\varphi:A \rightarrow B$ morfimo de anillos,
    la función $Spec(\varphi):Spec(B)\rightarrow Spec(A)$
    cumple que:
    \begin{enumerate}
        \item $Spec(\varphi)^{-1}(V(M))=V(\varphi(M))$.
        \item $V(\varphi^{-1}(\mathfrak{b}))=cl(Spec(\varphi)(V(\mathfrak{b})))$.
    \end{enumerate}
\end{proposicion}
\begin{proof}
    \begin{enumerate}
        \item Sea $\mathfrak{q}\in Spec(B)$
        \begin{enumerate}
            \item[$\subset$)] Si $\mathfrak{q}\in V(\varphi(M))$
            entonces $\varphi(M)\subset \mathfrak{q}$ por lo que 
            $$M\subset \varphi^{-1}(\varphi(M))\subset \varphi^{-1}(\mathfrak{q})$$
            asi que $Spec(\varphi)(\mathfrak{q})\in V(M)$
            $$\therefore \mathfrak{q}\in Spec(\varphi)^{-1}(V(M))$$.
            \item[$\supset$)] Si $\mathfrak{q}\in Spec(\varphi)^{-1}(V(M))$
            entonces $Spec(\varphi)(\mathfrak{q})\in V(M)$ y por ello,
            $M \subset Spec(\varphi)(\mathfrak{q}) = \varphi^{-1}(\mathfrak{q})$
            y entonces se sigue que 
            $$\varphi(M)\subset \varphi(\varphi^{-1}(\mathfrak{q}))\subset \mathfrak{q}$$
            asi que $\mathfrak{q}\in V(\varphi(M))$.
        \end{enumerate}
        \item Notemos que
        \begin{multline*}
            I(Spec(\varphi)(V(\mathfrak{b})))
            = I(\varphi^{-1}(V(\mathfrak{b})))
            = \bigcap_{\mathfrak{q}\in V(\mathfrak{b})}\varphi^{-1}(\mathfrak{q})\\
            = \varphi^{-1}(\bigcap_{\mathfrak{q}\in V(\mathfrak{b})}\mathfrak{q})
            = \varphi^{-1}(rad(\mathfrak{b}))
            = rad(\varphi^{-1}(\mathfrak{b}))
        \end{multline*}
        Esto gracias al lema \ref{lm:rad_conmuta}, asi que 
        \begin{multline*}
            cl(Spec(\varphi)(V(\mathfrak{b})))
            =V(I(Spec(\varphi)(V(\mathfrak{b}))))\\
            =V(rad(\varphi^{-1}(\mathfrak{b})))
            =V(\varphi^{-1}(\mathfrak{b}))
        \end{multline*}

    \end{enumerate}
\end{proof}

\begin{observacion}{}{}
    Observemos que la primer parte de 
    la proposición \ref{pr:continuidad} dice que $Spec(\varphi)$
    es una función continua.\\
    Notemos que en particular para $f\in A$
    \begin{multline*}
        Spec(\varphi)^{-1}(D(f))
        = Spec(\varphi)^{-1}(Spec(A)\setminus V(f))\\
        = Spec(B)\setminus Spec(\varphi)^{-1}(V(f))
        = Spec(B)\setminus V(\varphi(f))
        = D(\varphi(f))
    \end{multline*}

    Además si tenemos dos morfimos de anillos
    $f:A\rightarrow B$ y $g:B\rightarrow C$ 
    ocurre que $Spec(g\circ f)= Spec(f)\circ Spec(g)$,
    por lo que $Spec(-)$ define un funtore
    contravariante de la categoria de anillos a la de 
    espacios topológicos.
\end{observacion}

\begin{proposicion}{}{}
    Sean $A, B$ anillos y $\varphi:A \rightarrow B$ 
    morfismo de anillos entonces $Spec(\varphi)$ es un
    morfismo dominante si y solo si todos los elementos
    de Ker($\varphi$) son nilpotentes.
\end{proposicion}
\begin{proof}
    Por el lema \ref{pr:continuidad} se tiene que
    \begin{multline*}
        V(\mbox{Ker}(\varphi))=V(\varphi(\{0\}))
        =cl(Spec(\varphi)(V(0)))\\
        =cl(Spec(\varphi)(Spec(A)))
        =cl(\mbox{Im}(Spec(\varphi)))
    \end{multline*}
    \begin{align*}
        Spec(\varphi) \mbox{ es dominante }\iff\\
        cl(\mbox{Im}(Spec(\varphi))) = Spec(B) \iff \\
        V(\mbox{Ker}(\varphi)) = Spec(B) \iff\\
        \forall f\in \mbox{Ker}(\varphi),\quad V(f)=Spec(B)\iff \\
        \forall f\in \mbox{Ker}(\varphi),\quad D(f)=\varnothing \iff \\
        \forall f\in \mbox{Ker}(\varphi),\quad \mbox{f es nilpotente}.
    \end{align*}
    Esto último gracias a la proposición \ref{pr:nilpotentes}.
\end{proof}

Recordemos un lema de localizaciones

\begin{lema}{}{loc_iny}
    Sean $A$ un anillo, $S\subset A$ un conjunto multiplicativo
    y $I\subset S^{-1}A$ es un ideal, entonces
    $I= S^{-1}i^{-1}(I)$ donde
    $i:A\rightarrow S^{-1}A$ es el morfismo canónico.
\end{lema}
\begin{proof}
    \begin{enumerate}
        \item[$\subset$)] Sea $a/s\in I$ entonces, como $I$ es un ideal, 
        $i(a)=a/1=(s/1)(a/s)\in I$ asi que $a\in i^{-1}(I)$
        por lo que $a/s\in S^{-1}i^{-1}(I)$.
        \item[$\supset$)]Si $a/s\in S^{-1}i^{-1}(I)$ se 
        tiene que $a\in i^{-1}I$ y entonces $i(a)=a/1\in I$ 
        y como $I$ es un ideal $a/s=(1/s)(a/1)\in I$
    \end{enumerate}
\end{proof}


\begin{proposicion}{}{}
    Sea $A$ un anillo
    \begin{enumerate}
        \item Sea $\varphi:A\rightarrow B$ un epimorfismo de anillos
        entonces $Spec(\varphi)$
        es un homeomorfismo entre $Spec(B)$ y $V(\mbox{Ker}(\varphi))\subset Spec(A)$
        con su topología de subespacio.
        \item Sean $S\subset A$ un subconjunto multiplicativo y
        $i:A\rightarrow S^{-1}A$ el morfismo canónico, entonces
        $Spec(i)$ es un homeomorfismo entre $Spec(S^{-1}A)$ y 
        $\mathcal{S}=\{\mathfrak{p}\in Spec(A)\mid S\cap \mathfrak{p} = \varnothing\}$
    \end{enumerate}
\end{proposicion}
\begin{proof}
    \begin{enumerate}
        \item Veamos primero que si $\mathfrak{q}\in Spec(B)$
        entonces $\{0\}\subset \mathfrak{q}$ y por ende 
        $\mbox{Ker}(\varphi)=\varphi^{-1}(0)\subset\varphi^{-1}(\mathfrak{q}) 
        = Spec(\varphi)(\mathfrak{q})$
        $$\therefore Spec(\varphi)(\mathfrak{q}) \in V(\mbox{Ker}(\varphi))$$
        Como se quería, ahora veamos la biyectividad.\\
        Por el primer teorema de isomorfismo se tiene que $B \simeq A/\mbox{Ker}(\varphi)$
        y por el teorema de correspondencia hay una biyección entre ideales de $A/\mbox{Ker}(\varphi)$
        e ideales $I\subset A$ tales que $\mbox{Ker}(\varphi)\subset I$ y la biyección esta dada por
        tomar preimagen asi que en particular $Spec(\varphi)$ es la restricción de esta biyección
        a $V(\mbox{Ker}(\varphi))$ esta biyección es precisamente con $Spec(A/\mbox{Ker}(\varphi))$ que 
        por el isomorfismo esta en biyección con $Spec(B)$. Resumiendo el argumento,
        el teorema de correspondencia nos garantiza que $Spec(\varphi)$ es una biyección entre
        $V(\mbox{Ker}(\varphi))$ y $Spec(B)$.\\
        Finalmente si $V(\mathfrak{b})\subset Spec(B)$ es un cerrado se tiene que 
        \begin{align*}
            &Spec(\varphi)(V(\mathfrak{b}))\\
            &=\{Spec(\varphi)(\mathfrak{q})\mid \mathfrak{q}\in V(\mathfrak{b})\}\\
            &=\{ Spec(\varphi)(\mathfrak{q})\mid \mathfrak{b}\subset \mathfrak{q} \in Spec(B)\}\\
            &=\{\varphi^{-1}(\mathfrak{q})\mid \mathfrak{b}\subset \mathfrak{q}\in Spec(B)\}\\
            &= \{\varphi^{-1}(\mathfrak{q})\mid \varphi^{-1}(\mathfrak{b}) \subset \varphi^{-1}(\mathfrak{q})\in V(\mbox{Ker}(\varphi))\}\\
            &= \{\mathfrak{p}\in V(\mbox{Ker}(\varphi))\mid \varphi^{-1}(\mathfrak{b}) \subset \mathfrak{p}\}\\
            &= V(\mbox{Ker}(\varphi))\cap V(\varphi^{-1}(\mathfrak{b}))
        \end{align*}
        Por lo que $Spec(\varphi)$ es una función cerrada en su imagen.
        Concluimos que $Spec(\varphi)$ es un homeomorfismo en su imagen
        \item Recordemos que $i:A\rightarrow S^{-1}A$ esta dada por $i(a)=a/1$ 
        Primero notemos que dado $\mathfrak{p}\in Spec(S^{-1}A)$ entonces
        $Spec(i)(\mathfrak{p})\cap S= \varnothing$ pues de lo contrario
        existiria $s\in S\cap i^{-1}(\mathfrak{p})$, es decir, $i(s)\in \mathfrak{p}$
        pero $i(s)$ es una unidad de $S^{-1}A$ y entonces $\mathfrak{p}=S^{-1}A$!
        pero esto no es posible ya que al ser un ideal primo debe ser propio.
        Asi que $\mbox{Im}(Spec(i))\subset \mathcal{S}$.
        Probemos la biyectividad.
        \begin{enumerate}
            \item[$Sobre$] Sea $\mathfrak{p} \in \mathcal{S}$ y consideremos 
            $$S^{-1}\mathfrak{p}=\{p/s\mid p\in\mathfrak{p}, s\in S\}$$
            Como $p/s=1/1\iff \exists t\in S, t(p-s)=0\iff \exists t\in S, tp=ts\in S$
            entonces como $\mathfrak{p}$ es un ideal $tp\in \mathfrak{p}$ lo cual 
            contradice la suposición, asi que $1/1 \notin S^{-1}\mathfrak{p}$
            entonces es un subconjunto propio de $S^{-1}A$. Más aún,
            si $p_1/s_1, p_2/s_2\in S^{-1}\mathfrak{p}$ entonces
            $p_1/s_1 + p_2/s2 = (s_2p_1+s_1p_2)/(s_1s_2)\in S^{-1}\mathfrak{p}$
            y también $(p_1/s_1)(p_2/s_2)= (p_1p_2)/(s_1s_2)\in S^{-1}\mathfrak{p}$
            pues $\mathfrak{p}$ es un ideal. Finalmente si $a_1/s_1, a_2/b_2 \in S^{-1}A$
            son tales que $(a_1/b_1)(a_2/b_2)\in S^{-1}\mathfrak{p}$ entonces $a_1a_2\in\mathfrak{p}$
            por lo que $a_1\in \mathfrak{p}$ o $a_2\in \mathfrak{p}$ por lo tanto,
            $a_1/b_1 \in S^{-1}\mathfrak{p}$ o $a_2/b_2 \in S^{-1}\mathfrak{p}$, 
            por lo que $S^{-1}\mathfrak{p}\in Spec(S^{-1}A)$.\\
            Notemos que 
            \begin{multline*}
                Spec(i)(S^{-1}\mathfrak{p})=i^{-1}(S^{-1}\mathfrak{p})\\
                =\{a\in A\mid i(a)=a/1 \in \mathfrak{p}\}
                =\{a \in A\mid a\in \mathfrak{p}\} = \mathfrak{p}
            \end{multline*}
            Asi que $Spec(i)$ es sobre.
            \item[Inyectiva]Si $\mathfrak{p}, \mathfrak{q}\in Spec(S^{-1}A)$
            son tales que $Spec(i)(\mathfrak{p})=Spec(i)(\mathfrak{q})$, es decir,
            $i^{-1}(\mathfrak{p})= i^{-1}(\mathfrak{q})$ por el lema \ref{lm:loc_iny}
            se tiene que 
            $$\mathfrak{p}=S^{-1}i^{-1}(\mathfrak{p})
            =S^{-1}i^{-1}(\mathfrak{q})=\mathfrak{q}$$.
            Por lo que $Spec(i)$ es inyectiva.
        \end{enumerate}
        Y para terminar veamos que es una función cerrada ya que,
        si $V(\mathfrak{b})\subset Spec(S^{-1}A)$ es un cerrado, entonces:
        \begin{align*}
            &Spec(i)(V(\mathfrak{b}))\\
            &=\{Spec(i)(\mathfrak{q}) \mid \mathfrak{q}\in V(\mathfrak{b})\}\\
            &=\{i^{-1}(\mathfrak{q})\mid \mathfrak{b}\subset \mathfrak{q}\in Spec(S^{-1}A)\}\\
            &=\{i^{-1}(\mathfrak{q})\in \mathcal{S}\mid i^{-1}(\mathfrak{b})\subset i^{-1}(\mathfrak{q})\}\\
            &= \mathcal{S}\cap V(i^{-1}(\mathfrak{b}))
        \end{align*}
        Concluimos que $Spec(i)$ es un homeomorfismo entre 
        $\mathcal{S}$ y $Spec(S^{-1}A)$.
    \end{enumerate}
\end{proof}

\begin{ejemplo}{}{}
    Consideremos $\pi:\mathbb{Z}\rightarrow \mathbb{Z}/n\mathbb{Z}$
    la proyección canónica al cociente entonces
    $$\mbox{Ker}(\pi)=\{m\in\mathbb{Z}\mid \overline{m}=\overline{0}\}=\langle n\rangle$$
    y por el lema anterior 
    \begin{align*}
        Spec(\mathbb{Z}/n\mathbb{Z}) \simeq &V(n)\\
        &=\{\langle p\rangle\mid p\mbox{ es primo y existe }a\in \mathbb{Z}, n=pa\}\\
        &=\{\langle p_1\rangle, \dots, \langle p_l\rangle\}
    \end{align*}
    Donde $p_1, \dots, p_l$ son los factores primos de $n$.
    Además cada uno de esos puntos son cerrados en $Spec(\mathbb{Z})$
    entonces $Spec(\mathbb{Z}/n\mathbb{Z})$ es discreto y tiene
    tantos puntos como factores primos distintos tenga $n$.
\end{ejemplo}
\end{document}