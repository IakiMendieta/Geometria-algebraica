\documentclass[14pt]{extarticle}
\usepackage{amsthm}
\usepackage{amssymb}
\usepackage{xcolor}
\usepackage{tikz-page}
\usepackage{tikz,tkz-tab,amsmath}
\usepackage{tikz-cd}
\usepackage[many]{tcolorbox}
\usetikzlibrary{shadows.blur}
\usepackage[spanish]{babel}
\usepackage{geometry}
\geometry{
    a4paper,
    total={170mm,257mm},
    left=20mm,
    top=20mm,
    }

\newcommand{\catname}[1]{{\normalfont\textbf{#1}}}
\newcommand{\Set}{\catname{Set}}
\newcommand{\Ring}{\catname{Ring}}
\newcommand{\RModules}{\catname{R-Mod}}
\newcommand{\Ab}{\catname{Ab}}

\selectlanguage{spanish}
\tcbuselibrary{theorems}
\newtcbtheorem[number within=section]{teorema}{Teorema}%
{breakable,enhanced,colback=green!5,colframe=green!35!black,fonttitle=\bfseries}{th}
\newtcbtheorem[number within=section]{proposicion}{Proposición}%
{breakable,enhanced,colback=green!5,colframe=green!35!black,fonttitle=\bfseries}{pr}
\newtcbtheorem[number within=section]{corolario}{Corolario}%
{breakable,enhanced,colback=green!5,colframe=green!35!black,fonttitle=\bfseries}{cr}
\newtcbtheorem[number within=section]{lema}{Lema}%
{breakable,enhanced,colback=green!5,colframe=green!35!black,fonttitle=\bfseries}{lm}
\newtcbtheorem[number within=section]{definicion}{Definición}%
{breakable,enhanced,colback=red!5,colframe=red!35!black,fonttitle=\bfseries}{df}
\newtcbtheorem[number within=section]{observacion}{Observación}%
{breakable,enhanced,colback=blue!5,colframe=blue!35!black,fonttitle=\bfseries}{ob}
\newtcbtheorem[number within=section]{ejemplo}{Ejemplo}%
{breakable,enhanced,colback=blue!5,colframe=blue!35!black,fonttitle=\bfseries}{ej}

\tikzset{
        secnode/.style={
                minimum height=1cm,
                inner xsep=20pt,
                rotate=90,
                anchor=north east,
                draw=white,
                fill=violet,
                text=white,
                blur shadow={shadow blur steps=5,shadow blur extra rounding=1.3pt}},
        pagenode/.style={
                minimum width=5mm,
                minimum height=1cm,
                inner sep=2pt,
                anchor=south east,
                draw=white,
                fill=violet,
                text=white,
                blur shadow={shadow blur steps=5,shadow blur extra rounding=1.3pt}}
        }
\newcommand{\tikzpagelayout}{
        \draw[violet,line width=2pt,rounded corners=20pt] ([xshift=5mm]page.northwest) |- ([xshift=-2cm,yshift=5mm]page.southeast);
        \node[secnode] at (page.northwest) {Geometría algebraica. \thesection};
        \node[pagenode] at ([xshift=-1cm]page.southeast) {\thepage};
      }

\pagestyle{plain}

\title{Gavillas}
\author{Levia MN}
\date{\today}

\begin{document}
\maketitle
\section{Pregavillas}
\begin{definicion}{}{}
    Sea $(X, \tau)$ un espacio topológico,
    $\tau$ es un conjunto parcialmente ordenado
    por la contención $(\subset)$ podemos 
    considerar la categoria asociada $C_\tau$
    tal que sus elementos son los elementos de 
    $\tau$ y existe $V\rightarrow U$ si y solo si 
    $V \subset U$.\\
    Una pregavilla sobre $(X,\tau)$ es un funtor
    $\mathcal{F}:C_{\tau}^{op}\rightarrow C$
    con $C$ una categoría, las que nos interesan
    para los propositos de estas notas son las 
    categorias \Set, \Ab, \Ring, \RModules.\\
    Adicionalmente para nuestra notación pondremos
    $$\rho_{V}^{U}
    =\mathcal{F}(V\rightarrow U):\mathcal{F}(U) \rightarrow \mathcal{F}(V)$$
    Y los llamaremos los morfismos de restricción.\\
    Diremos que una pregavilla $\mathcal{F}$ es 
    una gavilla si para todo $U\in \tau$ y para toda
    cubierta abierta $\mathcal{C}=\{U_i\mid i\in I\}$ de $U$
    se cumple que dadas $s_i \in \mathcal{F}(U_i)$ 
    tales que para cualquier par de indices $i, j\in I$
    se tiene que:
    $$\rho_{U_i \cap U_j}^{U_i}(s_i)=\rho_{U_i \cap U_j}^{U_j}(s_j)$$
    Entonces existe una única $s\in \mathcal{F}(U)$
    tal que $\rho_{U_i}^{U}(s) = s_i$
\end{definicion}

\begin{ejemplo}{}{}
    Los ejemplos clásicos de gavillas tienen
    que ver con funciones que se restringen 
    a abiertos y preservan esa estructura.
    Algunos ejemplos clásicos son 
    $$\mathcal{C}(U)=\{f:U\rightarrow \mathbb{R} \mid f \mbox{ es continua.}\}$$
    Y si $X$ es una variedad suave sobre $k\in \{\mathbb{R}, \mathbb{C}\}$
    $$\mathcal{O}(U)=\{f:X\rightarrow k\mid f\mbox{ es una función suave.}\}$$
    como y en ambos ejemplos las restricciones estan dadas por
    $\rho_{V}^{U}(f)=f\circ i_{V}$.
    Con $i_{V}:V\rightarrow U$ la inclusión canónica,
    y es claro que ambas son gavillas sobre $X$.
\end{ejemplo}

\begin{definicion}{}{}
    Dada un espacio topológico $(X,\tau)$
    y una gavilla $\Gamma$ sobre $X$,
    dado $x\in X$ podemos 
    definir $M_x=\{(U,s) \mid U\in \tau, s\in \Gamma(U)\}$
    y tener la realción dada por $(U,s)\sim (V,t)$ si y solo si
    existe $W \in \mathcal{N}(x)$ tal que $W \subset U\cap V$
    y $\rho_{W}^{U}(s)= \rho_{W}^{V}(t)$.
\end{definicion}

\begin{proposicion}{}{}
    La realción $\sim$ anterior es una relación de 
    equivalencia.
\end{proposicion}
\begin{proof}
    \begin{enumerate}
        \item Notemos que $U\subset U=U\cap U$ y
        $\rho_{U}^{U}(s)=\rho_{U}^{U}(s)$
        por lo que $(U,s)\sim (U,s)$
        \item La definición es claramente simétrica.
        \item Si $(U,s)\sim (V,t)$ y $(V,t)\sim(W,r)$
        entonces tenemos $W_1,W_2\in \mathcal{N}(x)$
        tales que $W_1\subset U\cap V$ y 
        $W_2\subset V\cap W$ y además
        $\rho_{W_1}^{U}(s)=\rho_{W_1}^{V}(t)$
        y $\rho_{W_2}^{V}(t)=\rho_{W_2}^{W}(r)$.\\
        Consideremos $W_3 = W_1\cap W_2 \in\mathcal{N}(x)$
        y notemos que $W_3 \subset (U\cap V)\cap (V\cap W)\subset U\cap W$
        y también
        \begin{align*}
            \rho_{W_3}^{U}(s)
            =\rho_{W_3}^{W_1}(\rho_{W_1}^{U}(s))
            =\rho_{W_3}^{W_1}(\rho_{W_1}^{V}(t))
            =\rho_{W_3}^{V}(t)\\
            =\rho_{W_3}^{W_2}(\rho_{W_2}^{V}(t))
            =\rho_{W_3}^{W_2}(\rho_{W_2}^{W}(r))
            =\rho_{W_3}^{W}(r)
        \end{align*}
        Pues $\Gamma$ es un funtor. Por lo tanto
        $(U,s)\sim (W,r)$
    \end{enumerate}
\end{proof}

\begin{definicion}{}{}
    Dado $(X,\tau)$ espacio topológico y
    $\Gamma$ una pregavilla sobre $X$,
    para cada $x\in X$ definiremos
    $$\Gamma_x=M_x/\sim$$
    El tallo de la gavilla en $x$.
    Y notemos que en las categorias que nos 
    interesan 
    $$\Gamma_x= \lim_{U\in \mathcal{N}(x)}\Gamma(U)$$
    De hecho si la categoria donde $\Gamma$
    toma valores es alguna arbitraria, esta
    es la definición del tallo de la gavilla 
    sobre $x$.
\end{definicion}

\begin{observacion}{}{}
    Tenemos una función natural
    para cada $U\in \mathcal{N}(x)$,
    $\pi_x^{U}:\Gamma(U)\rightarrow \Gamma_x$
    que asigna $s \mapsto [U,s]$
    y escribimos $s_x = \pi_x^{U}(s)$ cuando
    es claro por el contexto.
\end{observacion}

\begin{ejemplo}{}{}
    Consideremos dos espacios topológicos
    $(X,\tau_X),(Y, \tau_Y)$ y una función 
    continua $p:Y\rightarrow X$.\\
    Para cada $U \in \tau_X$ definamos:
    $$\Gamma(U)=\{f:U\rightarrow Y \mid p\circ f= id_{U}, f \mbox{ es continua.} \}$$
    es decir, el conjunto de secciones continuas de $p$
    sobre $U$. Entonces $\Gamma$ es una gavilla sobre $f$
    con las restricciones usuales.\\
    Ya que si $f\in \Gamma(U)$ y $V\subset U$
    entonces para cualquier $x\in V$, entonces
    $p\circ(f\circ i_V)(x)=(p\circ f)\circ i_V(x)
    =id_U(x)=x$ por lo que $p\circ \rho_{V}^{U}(f) = id_V$
    asi que las restricciones estan bien definidas.
\end{ejemplo}

\section{Espacio étale}
Ahora veremos que en cierto sentido cualquier
gavilla es la gavilla de secciones continuas
sobre algún espacio.

\begin{definicion}{}{}
    Para un espacio topológico $(X,\tau)$
    y una pregavilla $\Gamma$ definiremos 
    $$\widetilde{\Gamma}=\bigsqcup_{x\in X}\Gamma_x$$
    Y para cada $U\in \tau$ y $s\in \Gamma(U)$
    $$\mathcal{V}(U,s)=\{(x, s_x)\mid x\in U\}$$
    Finalmente tenemos una función $p:\widetilde{\Gamma}\rightarrow X$
    tal que $p(x,[U,s])=x$
\end{definicion}

\begin{proposicion}{}{}
    $\mathcal{B}=\{\mathcal{V}(U,s)\mid U\in\tau, s\in \Gamma(U)\}$
    es una base para una topología sobre $\widetilde{\Gamma}$.
\end{proposicion}
\begin{proof}
    \begin{enumerate}
        \item Dado $(x,[U,s])\in \widetilde{\Gamma}$
        entonces $(x,[U,s])=(x,s_x)\in \mathcal{V}(U,s)$
        por lo que 
        $$\widetilde{\Gamma}=\bigcup \mathcal{B}$$
        \item Sea $(x,[W,r])\in \mathcal{V}(U,s)\cap \mathcal{V}(V,t)$
        entonces por definición $x \in U\cap V$,
        y además $[V,t]=t_x=[W,r]=s_x=[U,s]$
        por lo que existe $W_1\in \mathcal{N}(x)$
        $W_1\subset U\cap V$ tal que
        $h:=\rho_{W_1}^{U}(s)=\rho_{W_1}^{V}(t)\in \Gamma(W)$
        asi que consideremos $B = \mathcal{V}(W_1, h)\in \mathcal{B}$
        Notemos que como $x\in W_1$ y también, 
        $W_1\subset W_1=W_1\cap U$ y 
        $\rho_{W_1}^{W_1}(h)=id_{W_1}(h)=h=\rho_{W_1}^{U}(s)$
        por lo que $[W,r]=[U,s]=[W_1,h]=h_x$
        asi que $(x,[W,r])\in B$\\
        Finalmente dada $(y,h_y)\in B$ entonces ocurre que 
        $y\in W_1 \subset U\cap V$ por lo que $y\in U$ y 
        $y\in V$ y además $h_y=[W_1, h]=[U,s]=[V,t]$ en $\Gamma_y$, ya que
        $W_1\subset U ( V )$ y además $\rho_{W_1}^{U}(s)=h=\rho_{W_1}^{V}(t)$
        asi que $(y,h_y)\in \mathcal{V}(U,s)\cap \mathcal{V}(V,t)$
        por lo que $B \subset \mathcal{V}(U,s)\cap \mathcal{V}(V,t)$
    \end{enumerate}
    Esto prueba que $\mathcal{B}$ es una base 
    para una topología $\tilde{\tau}$ sobre $\widetilde{\Gamma}$
\end{proof}

\begin{definicion}{}{}
    A $(\widetilde{\Gamma},\tilde{\tau})$
    se le llama el espacio étale de la gavilla 
    $\Gamma$
\end{definicion}

\begin{proposicion}{}{}
    La función $p:\widetilde{\Gamma}\rightarrow X$
    definida arriba es continua y es un homeomorfismo 
    local.
\end{proposicion}
\begin{proof}
    Dado $U\in\tau$ se tiene que 
    $$p^{-1}(U)=\{(x,[V,t])\mid x\in U, t\in \Gamma(V)\}$$
    Así que dado $(x, [V,t])\in p^{-1}(U)$ 
    consideremos $W=U\cap V$, $h =\rho_{W}^{V}(t)$ 
    y $B=\mathcal{V}(W,h)$.
    Es fácil ver que $[V,t]=[W,h]$ en $\Gamma_x$ ya que 
    $W\subset W\cap V$ y $h = \rho_{W}^{W}(h)=\rho_{W}^{V}(t)$
    por lo que $(x,[V,t])\in B$.
    Dado $(y,h_y)\in B$ ocurre que $y\in W\subset U$
    por lo que $(y,h_y)\in p^{-1}(U)$.
    $$\therefore (x,[V,t])\in B\subset p^{-1}(U)$$
    Por lo que $p^{-1}(U)$ es abierto. Asi que $p$
    es una función continua.\\
    Más aún para cada $U\in\tau$ y $s\in \Gamma(U)$
    $p(\mathcal{V}(U,s))=U$ por lo que es una función abierta
    y al restringirla a cualquier abierto báisco tenemos 
    un homeomorfismo, como cada punto tiene un abierto básico 
    que lo contiene, $p$ es un homeomorfismo local.
\end{proof}

\begin{proposicion}{}{}
    Sea $s\in \Gamma(U)$ entonces la función
    $f:U\rightarrow \widetilde{\Gamma}$ dada 
    por $f(x)=(x,s_x)$ es una sección continua 
    de $p$. Más aún, si $\Gamma$ es una gavilla
    se vale el regreso
\end{proposicion}
\begin{proof}
    Primero, notemos que dado $x\in f^{-1}(\mathcal{V}(V,t))$ 
    entonces $(x,s_x)\in \mathcal{V}(V,t)$ entonces
    $x\in V$ y también $s_x=t_x$ asi que existe
    $W\in \mathcal{N}(x)$ tal que $W\subset U\cap V$
    tal que $\rho_{W}^{U}(s)= \rho_{W}^{V}(t)$
    asi que dado $y\in W$ como $W,U,V\in\mathcal{N}(y)$
    y ocurre lo anterior, entonces $s_y=t_y$,
    por lo que $f(y)=(y,s_y)=(y,t_y)\in \mathcal{V}(V,t)$.
    Concluimos que $x\in W\subset f^{-1}(\mathcal{V}(V,t))$, 
    así que $f$ es continua.
    Y es claro que $p\circ f(x)= p(x,s_s)=x$ asi que 
    si es una sección continua de $p$.

    Por otro lado, supongamos que $\Gamma$ es una gavilla
    y sea $\sigma:U \rightarrow \widetilde{\Gamma}$
    una sección continua de $p$ y escribamos $\sigma(x)=(x,\sigma_1(x))$.
    Sea $x\in U$ entonces, como $\mathcal{B}$ es una base, existen
    $V^x\in \tau$ y $t^x\in \Gamma(V)$ tal que
    $\sigma(x)\in \mathcal{V}(V^x,t^x)$ y por lo tanto
    $\sigma_1(x)=t^x_x$. Como para todo $x\in X$,
    $x\in V^x$ entonces $X = \bigcup_{x\in X}V^x$
    Veamos que $\{t^x\mid x\in X\}$ son compatibles.
    Sean $x,y\in X$ y $a\in V_x\cap V_y$ entonces,
    $t^x_a=\sigma_1(a)=t^y_a$ por lo que existe $W_a\in \mathcal{N}(a)$
    tal que $W_a\subset V_x\cap V_y$ y 
    \begin{equation*}
        \rho_{W_a}^{V_x}(t^x) = \rho_{W_a}^{V_y}(t^y)
    \end{equation*}
    Asi, tenemos a la familia $\{W_a\mid a\in V_x\cap V_y\}$
    la cual cubre a $V_x\cap V_y$ y a la familia
    $\{h_a:=\rho_{W_a}^{V_x}(t^x) = \rho_{W_a}^{V_y}(t^y)\in 
    \Gamma(W_a)\mid a\in V_x\cap V_y\}$
    y además ocurre que
    dados $a,b\in V_x\cap V_y$.
    \begin{align*}
        \rho_{W_a\cap W_b}^{W_a}(h_a)
        =\rho_{W_a\cap W_b}^{W_a}(\rho_{W_a}^{V_x}(t^x))\\
        =\rho_{W_a\cap W_b}^{V_x}(t^x)
        =\rho_{W_a\cap W_b}^{W_b}(\rho_{W_b}^{V_x}(t^x))\\
        =\rho_{W_a\cap W_b}^{W_b}(h_b)
    \end{align*}
    Y como $\Gamma$ es una gavilla entonces existe
    una única $t^{x,y}\in \Gamma (V_x\cap V_y)$ tal que 
    para cualquier $a\in V_x\cap V_y$ ocurre que:
    $\rho_{W_a}^{V_x\cap V_y}(t^{x,y})=h_a$
    Además, veamos que para cualquier $a\in V_x\cap V_y$
    $$\rho_{W_a}^{V_x\cap V_y}(\rho_{V_x\cap V_y}^{V_x}(t^x))= \rho_{W_a}^{V_x}(t^x)=h_a$$
    y como $t^{x,y}$ es único, se sigue que 
    $t^{x,y}=\rho_{V_x\cap V_y}^{V_x}(t^x)$.
    Pero también ocurre que:
    $$\rho_{W_a}^{V_x\cap V_y}(\rho_{V_x\cap V_y}^{V_y}(t^y))= \rho_{W_a}^{V_y}(t^y)=h_a$$
    $\therefore \rho_{V_x\cap V_y}^{V_x}(t^x)=t^{x,y}=\rho_{V_x\cap V_y}^{V_y}(t^y)$.
    Y de nuevo como $\Gamma$ es una gavilla
    existe $t\in \Gamma(U)$ tal que para cualquier
    $x\in X$,ocurre que $\rho_{V_x}^{U}(t)=t^x$.\\
    Ahora bien para cualquier $x\in X$,
    debe ocurrir que $[U,t]=[V_x,t^x]$ ya que
    $V_x\subset V_x\cap U$ y $\rho_{V_x}^{V_x}(t^x)=t^x=\rho_{V_x}^{U}(t)$,
    es decir, para toda $x\in X$, se tiene que $t_x=t^x_x=\sigma_1(x)$
    $$\therefore \sigma(x)= (x,t_x)$$
    Como se quería.
\end{proof}

Notemos que la definición de pregavilla
nos da una noción natural de morfismos de
pregavillas, es decir, transformaciones 
naturales entre los funtores.

\begin{lema}{}{}%
    Sean $(X,\tau)$ un espacio topológico,
    $\Gamma$ una gavilla sobre $X$, 
    $U\in \tau$ y dos elementos
    $\alpha,\beta \in \Gamma(U)$.
    Ocurre que $\alpha=\beta$
    si y solo si para cualquier $x\in U$
    $\alpha_x = \beta_x$.
\end{lema}%
\begin{proof}
    \begin{enumerate}
        \item[$\implies$)] Si $\alpha=\beta$
        entonces ocurre que para cualquier 
        $x\in U$ que 
        $$\alpha_x=[U,\alpha]=[U,\beta]=\beta_x$$
        \item[$\impliedby$)]Supongamos que para cualquier
        $x\in U$ ocurre que $\alpha_x=\beta_x$
        ahora entonces para cada $x\in U$ existe
        $W_x\subset U$ tal que 
        $$\rho_{W_x}^{U}(\alpha)= \rho_{W_x}^{U}(\beta)$$
        Asi que de $\{W_x\mid x\in U\}$ es una cubierta
        abierta de $U$ y además tenemos a la familia
        $\{h_x:=\rho_{W_x}^{U}(\alpha)\in \Gamma(W_x)\mid x\in U\}$
        notemos que dados $x, y \in U$ ocurre que:
        \begin{align*}
            \rho_{W_x\cap W_y}^{W_x}(h_x)
            =\rho_{W_x\cap W_y}^{W_x}(\rho_{W_x}^{U}(\alpha))
            =\rho_{W_x\cap W_y}^{U}(\alpha) \\
            =\rho_{W_x\cap W_y}^{W_y}(\rho_{W_y}^{U}(\alpha))
            =\rho_{W_x\cap W_y}^{W_y}(h_y)
        \end{align*}
        Por lo que esta familia es compatible y como
        $\Gamma$ es una gavilla, existe una única 
        $\gamma\in \Gamma(U)$ tal que 
        para cada $x\in U$ ocurre que 
        $$\rho_{W_x}^{U}(\gamma)=h_x=\rho_{W_x}^{U}(\alpha)= \rho_{W_x}^{U}(\beta)$$
        entonces se sigue que 
        $\alpha=\beta$
    \end{enumerate}
\end{proof}


\begin{teorema}{}{}
    Sean $(X,\tau)$ espacio topológico, 
    $\Gamma$ una gavilla sobre $X$ y 
    $(\widetilde{\Gamma},\tilde{\tau})$ su espacio étale,
    si $\mathcal{G}$ es la gavilla de secciones
    de $p:\widetilde{\Gamma}\rightarrow X$, entonces
    $\Gamma \simeq \mathcal{G}$.
\end{teorema}
\begin{proof}
    Para cada $U\in \tau$ definamos
    $\alpha_U:\Gamma(U)\rightarrow \mathcal{G}(U)$
    dada por $\alpha_U(s)(x)=(x,s_x)$ y veamos que 
    es un transformación natural.

    Sean $x\in V\subset U$ y $s\in \Gamma(U)$
    $$\alpha_V(\rho_{V}^{U}(s))(x)
    =(x, \rho_{V}^{U}(s)_x)
    =(x,s_x)=\alpha_U(s)\circ i_V (x)$$
    Pues $s_x=[U,s]=[V,\rho_{V}^{U}(s)]=\rho_{V}^{U}(s)_x$
    ya que $v\subset V\cap U$ y $\rho_{V}^{V}(\rho_{V}^{U}(s))=\rho_{V}^{U}(s)$
    por lo que $\alpha_U$ defininen las componentes de una transformación
    natural.

    Por el teorema anterior y gracias a que 
    $\Gamma$ es un gavilla, se tiene que cada 
    $\alpha_U$ es suprayectiva y además el lema anterior prueba que 
    $\alpha_U$ es inyectiva para cada 
    $U\in \tau$ y por tanto la familia $\alpha_U$ con
    $U\in \tau$ definen las componentes 
    de un isomorfismo natural.
\end{proof}

\section{Igualadores}

Caundo tenemos una pregavilla $\mathcal{F}$
podemos escribir la condición de ser una
gavilla en términos de unos morfismos en 
específicos.

\begin{definicion}{}{}
    Sea $(X,\tau)$ un espacio topológico,
    $\mathcal{F}$ una pregavilla sobre $X$,
    $U\in\tau$ y $\{U_i\mid i\in I\}$ una 
    cubierta abierta de $U$.
    Definamos los siguientes morfismos.
    \begin{align*}%
        \rho:&\mathcal{F}(U)\rightarrow \prod_{i\in I}\mathcal{F}(U_i)\\
        &s \mapsto \left(\rho_{U_i}^{U}(s)\right)_{i\in I}\\
        \sigma:&\prod_{i\in I}\mathcal{F}(U_i)\rightarrow \prod_{(i,j)\in I\times I}\mathcal{F}(U_i\cap U_j)\\
        &(s_i)_{i\in I} \mapsto \left(\rho_{U_i\cap U_j}^{U_i}(s_i)\right)_{(i,j)\in I \times I}\\
        \sigma':&\prod_{i\in I}\mathcal{F}(U_i)\rightarrow \prod_{(i,j)\in I\times I}\mathcal{F}(U_i\cap U_j)\\
        &(s_i)_{i\in I} \mapsto \left(\rho_{U_i\cap U_j}^{U_j}(s_j)\right)_{(i,j)\in I \times I}\\
    \end{align*}%
\end{definicion}
\begin{observacion}{}{}
    Notemos que con la notación antes dada 
    la pregavilla $\mathcal{F}$ es una gavilla
    si y solo si el diagrama:
    \begin{tikzcd}
        \mathcal{F}(U) \arrow{r}{\rho}
        & \prod_{i\in I}\mathcal{F}(U_i)
        \arrow[shift left]{r}{\sigma}
        \arrow[shift right]{r}[swap]{\sigma'}
        & \prod_{(i,j)\in I\times I}\mathcal{F}(U_i\cap U_j)
    \end{tikzcd}
    es un igualador.
\end{observacion}
\begin{proof}
    \begin{enumerate}
        \item[$\implies$)] Si $\mathcal{F}$ es una 
        gavilla entonces la composición
        \begin{align*}
            &\sigma \circ\rho(s)
            =\sigma\left(\left(\rho_{U_i}^{U}(s)\right)_{i\in I}\right)
            =\left(\rho_{U_i\cap U_j}^{U_i}(\rho_{U_i}^{U}(s))\right)_{(i,j)\in I\times I}\\
            &=\left(\rho_{U_i\cap U_j}^{U}(s)\right)_{(i,j)\in I\times I}
            =\left(\rho_{U_i\cap U_j}^{U_j}(\rho_{U_j}^{U}(s))\right)_{(i,j)\in I\times I}\\
            &=\sigma'\left(\left(\rho_{U_j}^{U}(s)\right)_{j\in I}\right)
            =\sigma'\circ\rho(s)
        \end{align*}
        Por lo que $\rho$ iguala a $\sigma, \sigma'$. 
        Además, dada una flecha tal que iguala a $\sigma,\sigma'$
        digamos $f:A\rightarrow \prod_{i\in I}\mathcal{F}(U_i)$
        entonces para cada $a\in A$ ocurre que $\{f(a)=(s(a)_i)_{i\in I}\}$ 
        es una familia de secciones de 
        la familia $\{U_i\mid i\in I\}$ tales que 
        $\rho_{U_i\cap U_j}^{U_i}(f(a))=\rho_{U_i\cap U_j}^{U_j}(f(a))$
        pues son las coordenadas de $\sigma\circ f(a)=\sigma'\circ f(a)$
        y como $\mathcal{F}$ es gavilla existe una única 
        $s(a)\in \mathcal{F}(U)$ tal que para cada $i\in I$
        se tiene que $\rho_{U_i}^{U}(s(a))=s(a)_i$.
        Podemos definir entonces la función $s:A\rightarrow \mathcal{F}(U)$
        dada por $s(a)=s(a)$ como arriba la existencia y únicidad garantizan
        que esta función esta bien definida y es única. Además $\rho\circ s= f$
        por definición, asi que que el diagrama dicho es un igualador.
        \item[$\impliedby$)] Si el diagrama es un igualador entonces
        dada una familia de funciones compatibles $(s_i)_{i\in I}$
        con $s_i\in \mathcal{F}(U_i)$ entonces podemos dar una flecha 
        $f:\{*\}\rightarrow \prod_{i\in I}\mathcal{F}(U_i)$ tal que $f(*)=(s_i)_{i\in I}$
        la compatibilidad nos dice que esta flecha iguala a $\sigma, \sigma'$
        por lo que existe una única flecha $s:\{*\}\rightarrow \mathcal{F}(U)$
        tal que $\rho\circ s= f$ llamaremos $s=s(*)$ entonces ocurre que
        $(s_i)_{i\in I}=f(*)=\rho(s(*))=\rho(s)=(\rho_{U_i}^{U}(s))_{i\in I}$
        es decir $s\in\mathcal{F}(U)$ es tal que 
        para cada $i\in I$ se tiene que 
        $\rho_{U_i}^{U}(s)=s_i$, la unicidad de la flecha 
        nos dice que esta sección de $\mathcal{F}(U)$ es única, por lo que
        concluimos que $\mathcal{F}$ es una gavilla.
    \end{enumerate}
\end{proof}

Esta condición nos habla del pegado de secciones
que puede ocurrir en cualquier categoria. Y más aún
nos dice que $\mathcal{F}(U)$ puede ser visto como 
el límite del diagrama formado por $\{U_i\cap U_j\mid (i,j)\in I\times I\}$.
Así que dada una base $\mathcal{B}$ de $X$ y un abierto $U$,
podemos considerar los elementos de $\mathcal{B}$ junto con sus intersecciones
y notar que $\mathcal{F}(U)=\lim_{U\supset B\in \mathcal{B}} \mathcal{F}(B)$
\end{document}